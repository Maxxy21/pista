% spell-checker:disable
\begin{otherlanguage}{italian}
\chapter*{Riassunto}

La valutazione delle startup pitch affronta sfide nel fornire un'analisi coerente e accessibile in diversi contesti imprenditoriali. Questa tesi presenta Pista, una piattaforma di valutazione basata su GenAI, e confronta le sue prestazioni con Winds2Ventures (W2V), un altro sistema di valutazione GenAI sviluppato da un team di startup in collaborazione con il supervisore della tesi. È stata condotta un'analisi statistica utilizzando 22 pitch di startup universitarie per comprendere come diverse approcci di valutazione GenAI performano quando valutano contenuti identici.

Pista è stata sviluppata come applicazione web full-stack utilizzando Next.js 15, database Convex, autenticazione Clerk e integrazione GPT-4. Il sistema valuta i pitch attraverso quattro dimensioni ponderate: Problem-Solution Fit (30\%), Business Model \& Market (30\%), Team \& Execution (25\%) e Pitch Quality (15\%). Pista supporta caricamenti di testo, elaborazione di file e trascrizione audio con tracciamento del progresso di valutazione in tempo reale. La piattaforma fornisce feedback strutturato con raccomandazioni specifiche per il miglioramento ed è distribuita su https://pista-app.vercel.app.

L'analisi comparativa rivela un accordo moderato tra i sistemi GenAI con un coefficiente kappa di Cohen di 0.505 e 77.3\% di accordo osservato quando si categorizzano i pitch come sotto la media, nella media o buoni. Pista ha ottenuto una media di 5.36 rispetto ai 5.20 di W2V, mostrando una differenza sistematica di 0.16 punti che indica approcci di valutazione distinti piuttosto che variazioni casuali. Entrambi i sistemi hanno dimostrato modelli coerenti, con Pista che fornisce valutazioni più ottimistiche mentre W2V applica criteri di valutazione più conservativi.

L'analisi delle prestazioni mostra che Pista fornisce valutazioni in 30-60 secondi a \$0.10-0.15 per valutazione con disponibilità 24/7. Il sistema gestisce la valutazione del Problem-Solution Fit in modo più efficace ma mostra limitazioni nella valutazione delle capacità del team e del potenziale di esecuzione. I pitch nel settore tecnologico hanno mostrato le maggiori differenze di punteggio tra i sistemi, mentre i pitch nel settore sanitario hanno dimostrato i tassi di accordo più alti.

La ricerca dimostra che diversi sistemi di valutazione GenAI portano caratteristiche e prospettive distinte alla valutazione delle startup. Questo confronto facilitato dal supervisore tra sistemi di valutazione GenAI mostra che l'approccio di punteggio coerente di Pista funziona bene per contesti educativi e scenari di screening iniziale, mentre i modelli di punteggio variati di W2V riflettono meglio i contesti di decisione degli investimenti. I risultati suggeriscono che multiple prospettive di valutazione GenAI forniscono una valutazione più completa rispetto al fare affidamento su singole piattaforme.

Questa tesi contribuisce con un confronto statistico documentato di piattaforme di valutazione GenAI utilizzando metriche standardizzate, un sistema proof-of-concept funzionante che dimostra la fattibilità tecnica, e prove empiriche delle capacità e limitazioni della valutazione GenAI. I risultati mostrano che i sistemi di valutazione GenAI possono fornire capacità preziose di valutazione delle startup mantenendo vantaggi pratici in velocità, efficienza dei costi e accessibilità per le comunità imprenditoriali sottorappresentate.

\end{otherlanguage}