% \chapter*{Abstract}
% Startup pitch evaluation has traditionally relied on human expert judgment, which creates inconsistency, scalability challenges, and accessibility barriers. This thesis presents Pista, a GenAI-powered system for automated startup pitch evaluation that addresses these limitations. The system uses GPT-4 to evaluate pitches across four dimensions: Problem-Solution Fit, Business Model \& Market, Team \& Execution, and Pitch Quality. It processes multiple input formats including text submissions and audio recordings.

% The evaluation compares Pista against Winds2Ventures, a commercial startup evaluation platform. Analysis of 22 startup pitches shows 27.3\% alignment within a one-point margin. Pista exhibits systematic optimism with an average 2.0-point positive bias compared to the commercial benchmark. The system demonstrates operational advantages including faster evaluation times (30-60 seconds), lower costs (\$0.10-0.15 per evaluation), and 24/7 availability.

% Dimensional analysis reveals varying performance across evaluation categories. Pista scores highest in Problem-Solution Fit (7.4) and lowest in Team \& Execution (6.7). Sectoral analysis shows technology pitches have the greatest score dispersion, while healthcare pitches achieve highest convergence (50.0\%).

% The implementation uses Next.js 15, React 18, Convex database, and Clerk authentication. The system supports real-time evaluation progress tracking and provides structured feedback with specific improvement recommendations. Performance optimization includes component lazy loading, edge runtime deployment, and comprehensive error handling.

% Current AI evaluation systems need significant calibration before approaching commercial platform reliability. The system faces inherent limitations. It relies on text transcription rather than visual cues, body language, and vocal intonation that human judges use. However, the results demonstrate AI's potential for scalable startup pitch screening. The system's consistency and accessibility advantages make it valuable for initial evaluation stages. This particularly benefits entrepreneurs in underrepresented regions with limited access to expert networks.

% Future enhancements could address current limitations. Interactive pitch sessions with real-time multimodal analysis would improve evaluation depth. AI-generated follow-up questions using video technology could simulate investor interactions. These developments would bridge the gap between current text-based evaluation and comprehensive human judgment. They would maintain operational advantages.