\chapter*{Abstract}
Startup pitch evaluation faces challenges in providing consistent, accessible assessment across diverse entrepreneurial contexts. This thesis presents Pista, a complete AI-powered evaluation platform that was built and deployed, and compares its performance with Winds2Ventures (W2V), another AI evaluation system developed under the same thesis supervisor. A statistical analysis was conducted using 22 university startup pitches to understand how different AI evaluation approaches perform when assessing identical content.

Pista was built as a full-stack web application using Next.js 15, Convex database, Clerk authentication, and GPT-4 integration. The system evaluates pitches across four weighted dimensions: Problem-Solution Fit (30\%), Business Model \& Market (30\%), Team \& Execution (25\%), and Pitch Quality (15\%). Pista supports text uploads, file processing, and audio transcription with real-time evaluation progress tracking. The platform provides structured feedback with specific improvement recommendations and is deployed at https://pista-app.vercel.app.

The comparative analysis reveals moderate agreement between the AI systems with a Cohen's kappa coefficient of 0.450 and 63.6\% observed agreement when categorizing pitches as below average, average, or good. Pista scored an average of 5.36 compared to W2V's 5.20, showing a systematic 0.17-point difference that indicates distinct evaluation approaches rather than random variation. Both systems demonstrated consistent patterns, with Pista providing more optimistic assessments while W2V applied more conservative evaluation criteria.

Performance analysis shows that Pista delivers evaluations in 30-60 seconds at \$0.10-0.15 per assessment with 24/7 availability. The system handles Problem-Solution Fit evaluation most effectively but shows limitations in assessing team capabilities and execution potential. Technology sector pitches showed the greatest scoring differences between systems, while healthcare pitches demonstrated the highest agreement rates.

The research demonstrates that different AI evaluation systems bring distinct characteristics and perspectives to startup assessment. This supervisor-facilitated comparison between AI evaluation systems shows that Pista's consistent scoring approach works well for educational contexts and initial screening scenarios, while W2V's varied scoring patterns better reflect investment decision contexts. The findings suggest that multiple AI evaluation perspectives provide more comprehensive assessment than relying on single platforms.

This thesis contributes the first documented statistical comparison of AI evaluation platforms using standardized metrics, a working proof-of-concept system demonstrating technical feasibility, and empirical evidence of AI evaluation capabilities and limitations. The results show AI evaluation systems can provide valuable startup assessment capabilities while maintaining practical advantages in speed, cost efficiency, and accessibility for underrepresented entrepreneurial communities.