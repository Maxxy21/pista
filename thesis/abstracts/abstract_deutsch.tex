% spell-checker:disable
\begin{otherlanguage}{ngerman}
\chapter*{Zusammenfassung}

Die Bewertung von Startup-Pitches steht vor Herausforderungen bei der Bereitstellung konsistenter und zugänglicher Bewertungen in verschiedenen unternehmerischen Kontexten. Diese Arbeit stellt Pista vor, eine GenAI-gestützte Bewertungsplattform, und vergleicht ihre Leistung mit Winds2Ventures (W2V), einem anderen GenAI-Bewertungssystem für Startups, das von einem Startup-Team in Zusammenarbeit mit dem Betreuer der Arbeit entwickelt wurde. Eine statistische Analyse wurde mit 22 universitären Startup-Pitches durchgeführt, um zu verstehen, wie verschiedene GenAI-Bewertungsansätze bei der Bewertung identischer Inhalte abschneiden.

Pista wurde als vollständige Webanwendung unter Verwendung von Next.js 15, Convex-Datenbank, Clerk-Authentifizierung und GPT-4-Integration entwickelt. Das System bewertet Pitches anhand von vier gewichteten Dimensionen: Problem-Solution Fit (30\%), Business Model \& Market (30\%), Team \& Execution (25\%) und Pitch Quality (15\%). Pista unterstützt Text-Uploads, Dateiverarbeitung und Audio-Transkription mit Echtzeit-Fortschrittsverfolgung der Bewertung. Die Plattform liefert strukturiertes Feedback mit spezifischen Verbesserungsempfehlungen und ist unter https://pista-app.vercel.app bereitgestellt.

Die vergleichende Analyse zeigt, dass die GenAI-Systeme moderate Übereinstimmung aufweisen, mit einem Cohen's Kappa-Koeffizienten von 0.505 und 77.3\% beobachteter Übereinstimmung bei der Kategorisierung von Pitches als unterdurchschnittlich, durchschnittlich oder gut. Pista erzielte einen Durchschnitt von 5.36 im Vergleich zu W2Vs 5.20, was eine systematische Differenz von 0.16 Punkten zeigt, die auf unterschiedliche Bewertungsansätze und nicht auf zufällige Variation hinweist. Beide Systeme zeigten konsistente Muster, wobei Pista optimistischere Bewertungen lieferte, während W2V konservativere Bewertungskriterien anwendete.

Die Leistungsanalyse zeigt, dass Pista Bewertungen mit 24/7-Verfügbarkeit in 30-60 Sekunden zu kosten von \$0.10-0.15 pro Bewertung. Das System handhabt die Problem-Solution Fit-Bewertung am effektivsten, zeigt aber Einschränkungen bei der Bewertung von Teamfähigkeiten und Ausführungspotential. Die Pitches im Technologiesektor zeigten die größten Bewertungsunterschiede zwischen den Systemen, während die Pitches im Gesundheitssektor die höchsten Übereinstimmungsraten demonstrierten.

Die Forschung zeigt, dass verschiedene GenAI-Bewertungssysteme unterschiedliche Eigenschaften und Perspektiven zur Startup-Bewertung bringen; dieser Vergleich, unterstützt vom Betreuer, zeigt, dass Pistas konsistenter Bewertungsansatz gut für Bildungskontexte und erste Screening-Szenarien funktioniert, während W2Vs variierte Bewertungsmuster besser Investitionsentscheidungskontexte widerspiegeln. Die Ergebnisse deuten darauf hin, dass mehrere GenAI-Bewertungsperspektiven eine umfassendere Bewertung bieten als das Vertrauen auf einzelne Plattformen.

Diese Arbeit trägt einen dokumentierten statistischen Vergleich von GenAI-Bewertungsplattformen unter Verwendung standardisierter Metriken bei, ein funktionierendes Proof-of-Concept-System, das technische Machbarkeit demonstriert, und empirische Belege für GenAI-Bewertungsfähigkeiten und -einschränkungen. Die Ergebnisse zeigen, dass GenAI-Bewertungssysteme wertvolle Startup-Bewertungsfähigkeiten bieten können, während sie praktische Vorteile in Geschwindigkeit, Kosteneffizienz und Zugänglichkeit für unterrepräsentierte unternehmerische Gemeinschaften beibehalten.

\end{otherlanguage}