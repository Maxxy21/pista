\chapter{Conclusion and Future Work}
\label{ch:conclusion}

This thesis presents Pista, a complete GenAI-powered startup pitch evaluation platform, and conducted a supervisor-facilitated statistical comparison with Winds2Ventures (W2V) to understand how different GenAI evaluation systems perform. Twenty-two university startup pitches were evaluated using both systems and the agreement patterns were analyzed using Cohen's kappa statistics.

\section{System Development and Key Findings}
\label{sec:accomplishments}

This section presents the system development outcomes and key research findings from this thesis work.

\textbf{Complete Functional System}: Pista was developed and deployed as a full-stack web application using Next.js 15, Convex database, and Clerk authentication. The system integrates GPT-4 for GenAI analysis across four evaluation dimensions. Pista supports text uploads, file processing, and audio transcription with real-time evaluation progress. The platform is live at https://pista-app.vercel.app and provides structured feedback with specific improvement recommendations.

\textbf{Statistical Comparison Results}: Moderate agreement was found between the two GenAI systems with a Cohen's kappa coefficient of 0.450 and 63.6\% observed agreement. Pista averaged 0.17 points higher than W2V (5.36 vs 5.20 out of 10), showing systematic differences in evaluation approaches. The statistical analysis reveals that different GenAI systems bring distinct perspectives to startup evaluation, even when analyzing identical pitch content.

\textbf{Practical Performance Insights}: Pista delivers evaluations in 30-60 seconds at \$0.10-0.15 per assessment, compared to traditional manual evaluation processes. The system provides 24/7 availability and consistent evaluation criteria across all pitches. The research found that GenAI evaluation works best for Problem-Solution Fit assessment but has limitations when evaluating team capabilities and execution potential.

\section{Research Contributions}
\label{sec:contributions}

This section outlines the contributions of this thesis to understanding GenAI-powered startup evaluation.

This work contributed a functional GenAI evaluation system with empirical comparison data demonstrating how different GenAI approaches perform in practice. The statistical analysis provides the first documented comparison of GenAI evaluation platforms using standardized metrics. Specific patterns were identified where GenAI evaluation excels and areas where it has limitations. The research demonstrates that GenAI evaluation systems can provide valuable assessment capabilities while maintaining practical advantages like speed, cost efficiency, and accessibility.

The deployed Pista system serves as a proof-of-concept for how modern GenAI technologies can be applied to startup evaluation challenges. The research demonstrated that building such systems is technically feasible and can produce meaningful results when properly designed and implemented.

\section{Practical Implications}
\label{sec:implications}

This section discusses how these findings apply to real-world startup evaluation scenarios.

The results show that GenAI evaluation systems like Pista work well for initial screening and educational contexts where consistent feedback helps entrepreneurs improve their pitches. The system's speed and accessibility make it valuable for competition organizers and startup accelerators processing many applications. However, the moderate agreement between systems suggests that multiple GenAI perspectives provide more comprehensive assessment than relying on a single platform.

The research found that different GenAI systems serve different purposes based on their evaluation characteristics. Pista's consistent scoring helps standardize feedback for learning environments. W2V's more varied scoring patterns better reflect investment decision contexts. Understanding these differences helps users choose appropriate GenAI tools for their specific evaluation needs.

\section{Limitations and Future Work}
\label{sec:limitations}

This section outlines the current limitations of this research and potential directions for future work.

\textbf{Current Limitations}: The evaluation used 22 university competition pitches, which represents a limited sample size and specific context. The comparison involved only two GenAI platforms, so broader patterns across multiple GenAI systems remain unknown. University pitches may not reflect the complexity and variety found in professional startup environments. These limitations mean the findings provide initial insights rather than comprehensive understanding of GenAI evaluation system performance.

\textbf{Technical Enhancement Opportunities}: Future versions of Pista could incorporate multimodal analysis including video and audio evaluation capabilities. Real-time interactive features where the GenAI asks clarifying questions could improve assessment depth. Integration with pitch presentation tools could provide seamless evaluation workflows. Enhanced machine learning models trained on larger datasets could improve evaluation accuracy and reduce score clustering patterns.

\textbf{Research Extensions}: Longitudinal studies tracking startup success rates could validate whether GenAI evaluation scores predict actual business outcomes. Comparative studies including more GenAI platforms would reveal broader patterns in automated evaluation approaches. Professional startup pitch datasets would test whether university pitch patterns apply to investment contexts. Human evaluator comparison studies could establish benchmarks for GenAI system performance against expert assessment.

\section{Final Reflections}
\label{sec:final-thoughts}

The development and evaluation of Pista demonstrates that creating functional GenAI evaluation systems is achievable with current technology and provides valuable insights into automated startup assessment. The statistical comparison with W2V reveals that GenAI evaluation systems bring distinct characteristics and perspectives to startup assessment, suggesting value in multiple-system approaches rather than relying on single platforms.

The moderate agreement between systems shows that GenAI evaluation captures meaningful patterns in pitch quality while maintaining practical advantages like speed, cost efficiency, and consistent availability. These capabilities make GenAI evaluation particularly valuable for initial screening, educational feedback, and accessibility in underrepresented regions where traditional evaluation resources may be limited.

This research shows that GenAI evaluation systems represent a practical tool for startup ecosystems, complementing rather than replacing human assessment. The key is understanding each system's strengths and limitations to use them effectively in appropriate contexts. As GenAI technology continues advancing, evaluation systems like Pista will likely become more sophisticated and valuable for supporting entrepreneurship development and startup assessment processes.