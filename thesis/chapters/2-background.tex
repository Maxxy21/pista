\chapter{Background and Related Work}
\label{ch:soa}

\section{Limitations of Traditional Pitch Evaluation Methods}
\label{sec:traditional-methods}

Startup pitch assessment has relied on industry expert judgment and intuition for decades. This happens primarily through expert panels or standardized frameworks. Both approaches remain integral to investor decision processes. However, research identifies three major problems: scalability, reliability, and accessibility. This section examines traditional expert panels and standardized frameworks. It focuses on challenges that GenAI-driven alternatives might address more effectively.

\subsection{Expert Panels}\label{subsec:expert-panels}
Expert panels composed of industry experts, investors, and entrepreneurs have dominated pitch evaluation for decades. These panels evaluate pitches through live presentations. They provide immediate feedback based on predetermined criteria regarding market opportunity and team capability. Supporters emphasize human insight's value in assessing characteristics difficult to quantify algorithmically, such as founder passion, confidence, and presentation effectiveness.

Despite human insight's value, research reveals significant consistency problems with expert panels. Gius (2024) analyzed data from 67 venture competitions and found that disagreement among expert judges evaluating identical pitches is common and actually predicts startup success \cite{Gius2024}. This assessment variability highlights concerns about the evaluation process's reliability and fairness..

Systematic bias presents another significant challenge. Balachandra et al. (2019) documented gender-based differences in investor evaluations. They showed that identical business propositions receive different treatment based on presenter demographics \cite{Balachandra2019}. Female entrepreneurs face more questions about risk mitigation strategies, while male entrepreneurs receive growth-focused questions. This indicates that even experienced evaluators exhibit unconscious biases.

\subsection{Standardized Frameworks}\label{subsec:standardized-frameworks}
To address subjectivity problems, many organizations have developed standardized evaluation criteria and scoring rubrics. These frameworks typically assess four main areas:
\begin{itemize}
    \item Market opportunity (market size, growth potential)
    \item Team capability (team experience, diversity)
    \item Product innovation (uniqueness, technical feasibility)
    \item Business model (revenue streams, scalability)
\end{itemize}

Organizations using structured rubrics show better inter-evaluator consistency compared to unstructured approaches \cite{Tsay2021VISUALSDI}.Despite this benefit, major challenges persist; Gompers et al. (2020) conducted extensive surveys of venture capital firms, revealing significant differences in evaluation practices. No standardized framework exists across the industry \cite{Gompers2020}. This variation suggests that while standardization addresses some bias, fundamental evaluation challenges remain.

Most standardized frameworks focus on these core evaluation dimensions. These appear in leading accelerator scoring rubrics and are supported by research as key startup success predictors \cite{Kalvapalle2024}.

\subsection{Key Limitations}\label{subsec:key-limitations}
Three interconnected challenges affect traditional evaluation processes:

\begin{enumerate}
    \item \textbf{Scalability}: Human evaluators face natural limits on pitch assessment volume. Hirshleifer et al. (2019) showed that financial analysts' forecast accuracy decreases as daily evaluation volume increases. Analysts make more basic decisions under cognitive fatigue \cite{Hirshleifer2019}. These findings suggest cognitive overload during busy evaluation periods reduces assessment quality.

    \item \textbf{Geographic Access Barriers}: Expert evaluators cluster in specific geographic locations. Colombo et al. (2019) documented how geographical distance affects startup funding patterns in Europe. Ventures distant from venture capital hubs are less likely to seek external equity due to perceived access barriers \cite{Colombo2019}. This geographic clustering creates challenges for startups in remote regions seeking quality evaluation.

    \item \textbf{Consistency Challenges}: Traditional assessment approaches struggle to consistently balance subjective and objective evaluation criteria. The relative importance of subjective characteristics versus objective metrics in predicting funding success varies inconsistently across evaluators \cite{Tsay2021VISUALSDI}.
\end{enumerate}

These challenges collectively disadvantage early-stage startups and founders from underrepresented backgrounds. This shows the need for more systematic and equitable evaluation approaches. Traditional evaluation methods face a fundamental tension between consistency and contextual judgment. GenAI systems can provide consistent criteria application and structured feedback. They may be uniquely positioned to address this tension.

\section{GenAI-Driven Evaluation Systems}
\label{sec:ai-systems}

Traditional assessment method limitations have driven growing interest in GenAI-based evaluation systems. These systems use advances in Natural Language Processing (NLP) and GenAI to address scalability, consistency, and accessibility challenges in pitch assessment.

\subsection{Current Approaches}\label{subsec:current-approaches}
Growing interest in GenAI-driven evaluation has produced several technological approaches. Each targets different aspects of traditional assessment challenges:

\begin{itemize}
    \item \textbf{Machine Learning-Based Assessment}: Researchers have developed systems that use historical startup data to inform evaluation decisions. Arroyo et al. (2019) showed machine learning methods' potential in venture capital decision support \cite{Arroyo2019}. These approaches analyze patterns in successful and unsuccessful ventures to identify key success indicators.

    \item \textbf{Text-Based Evaluation Systems}: Natural language processing techniques allow automated analysis of pitch content, business plans, and presentation materials. These systems can evaluate factors like clarity of value proposition, market understanding, and competitive positioning.

    \item \textbf{Commercial Evaluation Platforms}: Platforms like Winds2Ventures have emerged that combine GenAI-based evaluation tools with investor matching services. They attempt to provide integrated startup evaluation experiences \cite{w2v}. These platforms typically evaluate pitches across multiple dimensions and provide structured feedback to entrepreneurs.

    \item \textbf{Large Language Model Integration}: Recent advances in large language models enable more sophisticated text understanding and evaluation capabilities. These systems can process pitch content and provide detailed feedback across multiple evaluation dimensions.
\end{itemize}

\subsection{Performance Considerations}\label{subsec:performance-considerations}
GenAI systems offer several advantages over traditional methods:

\begin{itemize}
    \item \textbf{Scalability}: GenAI systems can process numerous pitches without cognitive fatigue that affects human evaluators. This enables consistent evaluation across high volumes.

    \item \textbf{Consistency}: Automated systems provide consistent evaluation criteria application, reducing variability observed in human expert panels.

    \item \textbf{Accessibility}: GenAI-based evaluation provides quality feedback to entrepreneurs regardless of geographic location. This addresses access barriers inherent in expert panel approaches.

    \item \textbf{Structured Feedback}: GenAI systems provide detailed, structured feedback across multiple evaluation dimensions. This helps entrepreneurs understand specific improvement areas.
\end{itemize}

However, important limitations remain. Human oversight remains necessary for contextual validation and final investment decisions \cite{Steyvers2024}. GenAI systems may struggle to understand nuanced aspects of entrepreneurship that require deep domain expertise or insights into market dynamics beyond textual analysis.

\subsection{Integration with Traditional Methods}\label{subsec:integration-with-traditional-methods}
Current research suggests that GenAI-driven evaluation systems serve fundamentally different functions than those of traditional human-centered approaches. Rather than direct competition, these systems demonstrate complementary capabilities that address distinct use cases within the evaluation ecosystem.

GenAI systems excel at providing consistent, scalable feedback for early-stage development and high-volume screening applications. Human experts remain essential for nuanced evaluation aspects requiring domain expertise, risk assessment, and investment decision-making. Understanding these systematic differences enables strategic deployment that maximizes each approach's strengths while mitigating their respective limitations.


