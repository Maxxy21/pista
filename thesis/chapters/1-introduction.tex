\chapter{Introduction}

\label{ch:introduction}

Communicating business ideas effectively can determine the success or failure of a startup. The startup pitch, as a primary vehicle for communicating visions, business models, and potential to investors and stakeholders, is traditionally evaluated through in-person presentations and expert judgment.  Research shows that investment decisions depend heavily on the quality of pitch presentations, which play a major role in securing financial support\cite{masterpresentat}. Many business functions have changed substantially through technological improvements. However, traditional pitch evaluation has not evolved much, still relying heavily on in-person presentations and individual expert judgment.



In recent years, GenAI, particularly large language models, has emerged as a powerful tool in business analysis. They have shown incredible abilities in natural language understanding, context, and structured evaluations in various domains \cite{Ozince2024}. Models such as GPT-4 represent the latest stage of large language models. They perform at levels comparable to humans in complex analytic tasks.This suggests a large potential to improve pitch evaluation processes through innovative approaches. \cite{gpt}.



However, pitch evaluation in the startup community continues to face several obstacles. Many evaluation processes use inconsistent standards. Quality feedback is hard to access, and evaluations take too long \cite{StartupEvaluati, Kalvapalle2024}. Current evaluation methods require significant time for each pitch and often yield vastly different interpretations among evaluators.These issues are especially experienced by entrepreneurs from emerging markets and other underrepresented groups, who have limited access to expert feedback networks. Expert feedback networks are limited in these areas \cite{BreakingBarrier}.



To address these challenges, this thesis develops Pista, a GenAI-based pitch evaluation system\footnote{Project Website, \url{https://startup-pitches.vercel.app}}, and compares it with an existing evaluation platform to understand the differences in their approaches.

The system processes pitches in multiple formats (text, files, audio recordings) and provides feedback across four dimensions: problem-solution fit, business model viability, team execution capability, and pitch quality. The comparison with Winds2Ventures reveals differences in scoring patterns and evaluation approaches.

The GenAI system focuses on textual content analysis, providing fast and consistent evaluation that can address accessibility barriers for entrepreneurs seeking feedback. The comparison reveals where GenAI evaluation works well and where traditional methods may be more suitable \cite{TheFutureofAIEv}.


In summary, this thesis develops a GenAI pitch evaluation system and compares it with an existing platform to understand their different strengths. The findings show how GenAI evaluation can provide accessible feedback for entrepreneurs while highlighting areas where traditional methods may be more appropriate \cite{Ozkazanc2022}.

% In summary, this project evaluates startup pitches by integrating Generative AI with human expert judgment. It aims to combine AI's rapid information evaluation capabilities with the nuanced insights of human evaluators. The GenAI system aims to support access and consistency for all founders, particularly those outside the main startup or entrepreneurial networks \cite{Ozkazanc2022}.