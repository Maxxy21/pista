\chapter{Introduction}

\label{ch:introduction}

Communicating business ideas effectively can determine the success or failure of a startup. The startup pitch, as a primary medium for communicating visions, business models, and potential to investors and stakeholders, is traditionally evaluated through in-person presentations and expert judgment.  Research shows that investment decisions are highly dependent on the quality of pitch presentations, which play an important role in securing financial support\cite{masterpresentat}. Many business functions have changed substantially through technological improvements. However, traditional pitch evaluation has not evolved much, still relying heavily on in-person presentations and individual expert judgment.

In recent years, GenAI, particularly large language models, has emerged as a powerful tool in business analysis. They have shown incredible abilities in natural language understanding, context, and structured evaluations in various domains \cite{Ozince2024}. Models such as GPT-4 represent the latest stage of large language models. They perform at levels comparable to humans in complex analytic tasks. This suggests a great potential to improve pitch evaluation processes through innovative approaches\cite{gpt}.

However, pitch evaluation in the startup community continues to face several obstacles. Many evaluation processes use inconsistent standards. Quality feedback is hard to access, and evaluations take too long \cite{StartupEvaluati, Kalvapalle2024}. Current evaluation methods require significant time for each pitch and often yield vastly different interpretations among evaluators. These issues are especially experienced by entrepreneurs from emerging markets and other underrepresented groups, who have limited access to expert feedback networks \cite{BreakingBarrier}.

To address these challenges, Pista, a GenAI-based pitch evaluation system \footnote{Project Website, \url{https://pista-app.vercel.app}}, is presented in this thesis and compares it with an existing Winds2Ventures (W2V)\footnote{\url{https://w2v.network/}}; another GenAI evaluation startup evaluation system developed by a startup team that has collaboration with the thesis supervisor.
The differences in their approaches are examined, and their performance is evaluated. The system processes pitches in multiple formats (text, files, audio recordings) and provides feedback across four dimensions: problem-solution fit, business model viability, team execution capability, and pitch quality. The comparison with Winds2Ventures reveals differences in scoring patterns and evaluation approaches.
The GenAI system focuses on textual content analysis, providing fast and consistent evaluation that can address accessibility barriers for entrepreneurs seeking feedback.  The comparison shows how different GenAI evaluation methods perform and where each approach works best \cite{TheFutureofAIEv}.

