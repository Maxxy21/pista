\chapter{Conclusion and Future Work}
\label{ch:conclusion}

This thesis developed Pista, a GenAI-powered startup pitch evaluation system, and compared its performance with Winds2Ventures (W2V). Through evaluation of 22 startup pitches, differences in scoring patterns and evaluation approaches were analyzed.

\section{Key Findings}
\label{sec:key-findings}

\textbf{Systematic GenAI Optimism}: Pista scored higher than Winds2Ventures in 100\% of cases (22/22 pitches), averaging +1.70 points difference. This universal bias indicates fundamental evaluation philosophy differences.

\textbf{Limited Discrimination}: 66.7\% of GenAI evaluations (15/22 pitches) received identical 7.0/10 scores, indicating critical limitations in quality differentiation. Commercial platforms demonstrated broader score distribution (4.3-6.4 range) with superior discrimination capabilities.

\textbf{Dimensional Variations}: GenAI systems performed strongest in Problem-Solution Fit evaluation (7.1/10) but weakest in Team \& Execution assessment (6.6/10), reflecting limitations in evaluating human factors and leadership capabilities.

\section{Deployment Implications}
\label{sec:applications}

The comparison revealed different strengths for each approach. Pista offers fast, consistent evaluation suitable for educational contexts and initial screening. W2V provides more varied scoring that better reflects real-world investment assessment needs.

\section{Research Contributions}
\label{sec:contributions}

This work developed a functional GenAI evaluation system and compared it with an existing evaluation platform. The comparison revealed systematic differences in scoring patterns and highlighted areas where GenAI evaluation shows promise and limitations.

\section{Limitations and Future Work}
\label{sec:limitations}

\textbf{Study Limitations}: The limited sample size (22 pitches) and university competition context may not fully represent professional investment scenarios. Single commercial platform comparison constrains broader generalizability.

\textbf{Future Research}: Larger professional samples, multi-platform analysis, and longitudinal tracking of startup outcomes would strengthen validation. Multimodal capabilities integrating video and audio analysis could address current text-only limitations. Interactive evaluation systems with GenAI-generated follow-up questions would bridge the gap between static analysis and dynamic human assessment.

\section{Conclusion}
\label{sec:conclusion}

The comparison shows that GenAI evaluation systems like Pista can provide useful feedback for entrepreneurs and educational settings, while platforms like W2V offer more nuanced assessment for investment contexts. Each approach has distinct strengths that make them suitable for different use cases.

Future work could focus on improving GenAI evaluation discrimination and reducing scoring convergence while maintaining the speed and accessibility advantages that make such systems valuable for entrepreneurship education and initial pitch development.